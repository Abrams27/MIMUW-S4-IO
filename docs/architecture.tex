%& --translate-file latin2pl
\documentclass{article}

\usepackage{polski}
\usepackage{hyperref}
\usepackage[T1]{fontenc}
\usepackage[utf8]{inputenc}
\usepackage{listings}
\hypersetup{
colorlinks=true,
urlcolor=blue,
}
\setcounter{section}{-1}


\title{
Lekizapteki\\
\large architektura}
\author{Marcin Abramowicz \and Mateusz Danowski \and Dawid Jamka \and Tomasz Patyna}


\begin{document}
  \maketitle

  \section{Dziennik zmian}
  \begin{tabular}{|c|c|c|}
    Nr iteracji & Data & Opis zmian \\
    \hline
    1. & 25.03.2020 & Utworzenie dokumentu oraz jego pierwsza wersja. \\
  \end{tabular}

  \section{Ogólna struktura}
  Aplikacja składa się z frontendu WWW napisanego w
  \href{https://angular.io}{Angularze}, backendu napisanego w
  \href{https://spring.io}{Spring Framewok} oraz bazy danych
  \href {https://spring.io}{H2}.

  \section{Komunikacja}
  \subsection{Frondend - Backend}
  Komunikacja odbywa się protokołem HTTP, serwer dostarcza
  \href{https://en.wikipedia.org/wiki/Representational_state_transfer}{RESTful API},
  obsługując rządania w formacie
  \href{https://en.wikipedia.org/wiki/JSON}{json}.
  Serwer jest dostepny na
  \href{http://students.mimuw.edu.pl:7312}{serwerze Students na porcie `7312`},
  natomiast frontend wykonuje zapytania za pomocą
  \href{https://angular.io/guide/http}{httpClient'a}.

  \subsubsection{Endpointy}
  \begin{lstlisting}
    GET /lekizapteki/diseases
    Accept: application/json

    Response:
    HTTP/1.1 200 (OK)
    Content-Type: application/json
    Body:
    [
      {
        "id": Long,
        "name":"String"
      }
    ]


    GET /lekizapteki/medicines
    Accept: application/json

    Parameters:
    diseaseId: Long (optional)

    Responses:
    HTTP/1.1 200 (OK)
    Content-Type: application/json
    Body:
    [
      {
        "ean": "String",
        "name": "String",
        "dose": "String"
      }
    ]

    HTTP/1.1 404 (Not Found)
    Content-Type: application/json
    Message: "No medicine on such disease"

    TODO?


    GET /lekizapteki/medicines/identical
    Accept: application/json

    Parameters:
    ean: String
    diseaseId: Long (optional)

    Responses:
    HTTP/1.1 200 (OK)
    Content-Type: application/json
    Body:
    [
      {
        "ean": "String",
        "name": "String",
        "dose": "String"
      }
    ]

    HTTP/1.1 404 (Not Found)
    Content-Type: application/json
    Message: "No medicine with such EAN"

    TODO

  \end{lstlisting}


\end{document}
