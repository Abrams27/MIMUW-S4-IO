%& --translate-file latin2pl
\documentclass{article}

\usepackage{polski}
\usepackage{hyperref}
\usepackage[T1]{fontenc}
\usepackage[utf8]{inputenc}
\hypersetup{
colorlinks=true,
urlcolor=blue,
}
\setcounter{section}{-1}


\title{
Lekizapteki\\
\large wymagania funkcjonalne}
\author{Marcin Abramowicz \and Mateusz Danowski \and Dawid Jamka \and Tomasz Patyna}


\begin{document}
  \maketitle

  \section{Dziennik zmian}
  \begin{tabular}{|c|c|c|}
    Nr iteracji & Data & Opis zmian \\
    \hline
    1. & 23.03.2020 & Utworzenie dokumentu oraz jego pierwsza wersja. \\
    \hline
    2. & 21.04.2020 & Poprawki, uszczegółowienie wymagań.
  \end{tabular}

  \section{Opis}
  Użytkownik ma do dyspozyji aplikację WWW, która umożliwia mu wyszukiwanie leków identycznych do szukanego na daną jednostkę chorobową.
  Serwis udostępnia przeglądanie i wyszukiwanie tylko leków, które są w niżej opisanej postaci.

  \section{Wymagania funkcjonalne}
    \subsection{Rozpatrywane postacie leków}
    TODO, trzeba z excelka wybrac jakie wgl dokladnie sa tam leki xd

    \subsection{Opis komponentu dalej nazywanego ``TODO''}
    Komponent umożliwia wybieranie elementu ze zbioru elementów.
    Jest on polem, na które można kliknąć.
    Po kliknięciu rozwija się bezpośrednio pod nim lista, natomiast początkowe pole, staje się polem umożliwiającym wpisywanie tekstu.
    Elementy w liście są posortowane alfabetycznie rosnąco (`a` jest przed `z`).
    Wybór elemenetu odbywa się poprzez wybranie go z listy, poprzez przewijanie jej i kliknięcie na element.
    Pole wejsciowe natomiast umożliwia filtrowanie elementów wyświetlanych w liście, są w niej wyświetlane jedynie pozycje zawierające wpisaną frazę.

    \subsection{Opis użytkowania}
      \subsubsection{Wybór jednostki chorobowej}
      Użytkownik wybiera chorobę z TODO.
      Nastepnie, zatwierdza wybór wciśnięciem guzika ``Zatwierdź chorobę''.
      Guzik znajduje się obok komponentu do wyboru choroby,
      Powoduje to pojawienie się komponent TODO do wybierania leku.

      \subsubsection{Wybór leku}
      Użytkownik może wyszukiwać lek podobnie jak chorobę lub po numerze EAN.
      Wybór ten dokonuje przesuwanym guzikem ``Szukaj po numerze EAN''.

      Jeśli nie wybierze tej opcji to wybiera lek, sposród tych które są przypisywane na wybraną jednostkę chrobową.
      Używa do tego również TODO.

      Jeśli zdecyduje się na wybieranie po numerze EAN, po przełączeniu guzika znika komponent TODO, a w jego miejsce pojawia się pole do wpisywania numeru EAN.

      Użytkownik wybór potwierdza nacisnięciem guzika ``szukaj''.

      Pojawia się po tym tableka zawierające leki generyczne do podanego, czyli takie które zawierają tę samą substancję czynną
      (o lekach generycznych można poczytać \href{https://pl.wikipedia.org/wiki/Lek_generyczny}{tutaj}).

      Pozycje są posortowane po dalej zdefiniowanej ``opłacalności''.
      Leki o takiej samej wartości``opłacalności' są wyświetlane posortowane alfabetycznie rosnąco.
      Są one wyświetlane w tabelce, a użytkownik może je przewijać.
      Każda pozycja zawiera nazwę leku, substancję czynną i cenę detaliczną.
      Po kliknięciu na wiersz w tabeli, wysuwa się pod nim panel z inforamcją dotyczącą numeru EAN tego leku, dawką oraz postacią.
      Pokazywane są również szczegóły dotyczące ceny oraz poziomu odpłatności, tj.
      `Urzędowa cena zbytu',
      `Cena hurtowa brutto',
      `Cena detaliczna', < do sprawdzenia
      `Wysokość limitu finansowania',
      `Poziom odpłatności' i
      `Wysokość dopłaty świadczeniobiorcy'.

      Jeżeli jednak użytkownik wyszukiwał po numerze EAN i podany numer jeat nieprawidłowy, zostanie wyświetlona informacja z napisem  ``Nieprawidłowy numer EAN''.
      Natomiast, jeżeli lek o danym numerze EAN istnieje, ale nie jest on przypisywany na wybraną jednostkę chorobową,
      wyświetli się informacja z napisem  ``Lek o podanym numerze EAN nie jest przypisywany na wybraną jednostkę chorobową''.

      \subsubsection{Definicja ``opłacalności''}
      Definiujemy jaką jako współczynnik ceny (chyba detalicznej TODO) do pojedynczej dawki leku.
      Na przykład, opakowanie leku ``LekABC'' zawiera 69 tabletek i kosztuje 12,37, wtedy współczynnik jest równy $\frac{12,37}{69} \approx 0,179$.
      Z kolei w opakowaniu ``LekXYZ'' są 42 tabletki i można go nabyć za 17,23, więc współczynnik to $\frac{17,23}{42} \approx 0.410$.
      Zakładając, że są identyczne, w wyniku wyszukiwania ``LekABC'' wyświetliłby się nad ``LekXYZ'', gdyż ma mniejszy współczynnik,
      co jest równoważne z mniejszą ceną za jedną tabletkę.

  \section{Wymagania niefunkcjonalne}

  Aplikacja dzięki swojej modułowości umożliwia łatwe dodawanie nowych funkcjonalności.
  Dodatkowo umożliwia prostą zmianę między żródłami zasilającymi aplikację (wczytanie nowego rozporządzenia na kolejny okres).


\end{document}
