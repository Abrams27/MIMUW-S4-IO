%& --translate-file latin2pl
\documentclass{article}

\usepackage{polski}
\usepackage{hyperref}
\usepackage[T1]{fontenc}
\usepackage[utf8]{inputenc}
\hypersetup{
colorlinks=true,
urlcolor=blue,
}
\setcounter{section}{-1}


\title{
Lekizapteki\\
\large wymagania funkcjonalne}
\author{Marcin Abramowicz \and Mateusz Danowski \and Dawid Jamka \and Tomasz Patyna}


\begin{document}
  \maketitle

  \section{Dziennik zmian}
  \begin{tabular}{|c|c|c|}
    Nr iteracji & Data & Opis zmian \\
    \hline
    1. & 23.03.2020 & Utworzenie dokumentu oraz jego pierwsza wersja. \\
  \end{tabular}

  \section{Opis}
  Użytkownik ma do dyspozyji aplikacje WWW, która umożliwia mu wyszukiwanie leków identycznych do szukanego na dana jednostke chorobową.

  \section{Wymagania funkcjonalne}
  Użytkownik wybiera chorobę, a następnie, na podstawie wybranej jednostki chorobowej, lek.
  Przy wyborze leku ma możliwość wybrania, czy szuka na podstawie nazwy leku, czy numeru EAN.

  Do wyboru chroby i leku na podstawie jego nazywy użytkownik robi to z rozwijanej listy, w której elementy są posortowane alfabetycznie.
  Znajduje się nad nią pole umożliwijące wyszukiwanie pozycji w liście, które zawierają podawaną frazę.

  Jeśli użytkownik zdecyduje się na wyszukiwanie leku po numerze EAN, zamiast listy pojawia się pole na wpisanie numeru.

  Po wciśnięciu guzika z napisem ``szukaj'' otrzymuje listę leków na daną chorobę, które są identyczne do podanego.
  Pozycje są posortowanie po dalej zdefiniowanej ``opłacalności''.
  Są one wyświetlane w tablece, a użytkownik może je przewijać.
  Każda pozycja zawiera nazwę leku, substancję czynną i cenę detaliczną.
  Po kliknięciu na wiersz w tabeli, wysuwa się pod nim panel z inforamcją dotyczącą numeru EAN tego leku, dawką oraz postacią.
  Pokazywane są również szczegóły dotyczące ceny oraz poziomu odpłatności, tj. ... (tu beda podane dokaldne pola)

  Jeżeli użytkownik wybrał opcję wyszukiwania leku po numerze EAN i podał nieprawidowy numer zostanie wyświetlona informacja z napisem
  ``Nieprawidłowy numer EAN''.
  Jeżeli lek o danym numerze EAN istnieje, ale nie jest on przypisywany na wybrana jednostkę chorobową, wyświetli się informacja z napisem
  ``Lek o podanym numerze EAN nie jest przypisywany na wybraną jednostkę chorobową''.

  ``Opłacalność'' definiujemy jako współczynnik ceny do pojedyńczej dawki leku, np. ... (tu jakis przyklad, cena leku na tabletki?)

  \section{Wymagania niefunkcjonalne}

  ... (Do wymyslenia)

\end{document}
