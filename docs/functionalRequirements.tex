%& --translate-file latin2pl
\documentclass{article}

\usepackage{polski}
\usepackage{hyperref}
\usepackage[T1]{fontenc}
\usepackage[utf8]{inputenc}
\hypersetup{
colorlinks=true,
urlcolor=blue,
}
\setcounter{section}{-1}


\title{
Lekizapteki\\
\large wymagania funkcjonalne}
\author{Marcin Abramowicz \and Mateusz Danowski \and Dawid Jamka \and Tomasz Patyna}


\begin{document}
  \maketitle

  \section{Dziennik zmian}
  \begin{tabular}{|c|c|c|}
    Nr iteracji & Data & Opis zmian \\
    \hline
    1. & 23.03.2020 & Utworzenie dokumentu oraz jego pierwsza wersja. \\
  \end{tabular}

  \section{Opis}
  Użytkownik ma do dyspozyji aplikację WWW, która umożliwia mu wyszukiwanie leków identycznych do szukanego na daną jednostkę chorobową.
  Serwis udostępnia przeglądanie i wyszukiwanie tylko leków, które są w postaci tabletek.

  \section{Wymagania funkcjonalne}
  Użytkownik wybiera chorobę.
  Nastepnie, po zatwierdzeniu wyboru guzikiem ``Zatwierdź chorobę'' pojawia się pole do wybierania leku.
  Użytkownik wybiera lek, sposród tych które są przypisywane na tą jednostkę chrobową.

  Przy wyborze leku ma możliwość zdecydowania, czy szuka na podstawie nazwy leku, czy numeru EAN.

  Do wyboru choroby i leku na podstawie jego nazwy, użytkownik korzysta z rozwijanej listy, w której elementy są posortowane alfabetycznie.
  Znajduje się nad nią pole umożliwijące wyszukiwanie pozycji w liście, które zawierają podawaną frazę.

  Do zmiany sposobu szukania leku służy przełącznik, po naciśnięciu którego zamiast listy pojawia się pole na wpisywanie numeru.
  Przełącznik umożliwia również powrót do wyszukiwania po nazwie.

  Po wciśnięciu guzika z napisem ``szukaj'' otrzymuje listę leków na daną chorobę, które są identyczne do podanego.
  Pozycje są posortowane po dalej zdefiniowanej ``opłacalności''.
  Są one wyświetlane w tabelce, a użytkownik może je przewijać.
  Każda pozycja zawiera nazwę leku, substancję czynną i cenę detaliczną.
  Po kliknięciu na wiersz w tabeli, wysuwa się pod nim panel z inforamcją dotyczącą numeru EAN tego leku, dawką oraz postacią.
  Pokazywane są również szczegóły dotyczące ceny oraz poziomu odpłatności, tj.
  `Urzędowa cena zbytu',
  `Cena hurtowa brutto',
  `Cena detaliczna',
  `Wysokość limitu finansowania',
  `Poziom odpłatności' i
  `Wysokość dopłaty świadczeniobiorcy'.

  Jeżeli użytkownik wybrał opcję wyszukiwania leku po numerze EAN i podał nieprawidłowy numer, zostanie wyświetlona informacja z napisem
  ``Nieprawidłowy numer EAN''.
  Jeżeli lek o danym numerze EAN istnieje, ale nie jest on przypisywany na wybraną jednostkę chorobową, wyświetli się informacja z napisem
  ``Lek o podanym numerze EAN nie jest przypisywany na wybraną jednostkę chorobową''.

  Jako leki indetyczne rozumiemy leki, które są
  \href{https://www.eupati.eu/pl/rodzaje-lekow/leki-generyczne/}{generyczne} do oryginalnego.

  ``Opłacalność'' definiujemy jako współczynnik ceny do pojedynczej dawki leku.
  Na przykład, opakowanie leku ``LekABC'' zawiera 69 tabletek i kosztuje 12,37, wtedy współczynnik jest równy $\frac{12,37}{69} \approx 0,179$.
  Z kolei w opakowaniu ``LekXYZ'' są 42 tabletki i można go nabyć za 17,23, więc współczynnik to $\frac{17,23}{42} \approx 0.410$.
  Zakładając, że są identyczne, w wyniku wyszukiwania ``LekABC'' wyświetliłby się nad ``LekXYZ'', gdyż ma mniejszy współczynnik,
  co jest równoważne z mniejszą ceną za jedną tabletkę.

  \section{Wymagania niefunkcjonalne}

  Aplikacja dzięki swojej modułowości umożliwia łatwe dodawanie nowych funkcjonalności.
  Dodatkowo umożliwia prostą zmianę między żródłami zasilającymi aplikację (wczytanie nowego rozporządzenia na kolejny okres).


\end{document}
