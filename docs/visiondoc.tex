%& --translate-file latin2pl
\documentclass{article}

\usepackage{polski}
\usepackage{hyperref}

\title{MEME TO (FAJFUS) APP}
\author{Marcin Abramowicz \and Mateusz Danowski \and Dawid Jamka \and Tomasz Patyna}

\begin{document}
\maketitle


\section{Motywacja}

Istnienie systemu refundacji leków umożliwia łatwiejszy i tańszy dostęp pacjentów do substancji leczniczych. W celu usystematyzowania całego procesu Ministerstwo Zdrowia cyklicznie udostępnia dokumenty opisujące ceny oraz wartości refundacji poszczególnych leków, których koszta Państwo postanawia w pełni bądź częściowo finansować. Liczba zawartych w nich pozycji może być przytłaczająca dla lekarzy przypisujących te środki, farmaceutów oraz pacjentów. Stając naprzeciw ich potrzebom potrzebna jest aplikacja operująca na tych zbiorach danych umożliwiająca analizowanie ich oraz szybkie i intuicyjne przeszukiwanie.


\section{Funkcjonalność}

Aplikacja WWW bazuje na danych udostepnianych przez
\href{https://www.gov.pl/web/zdrowie/obwieszczenia-ministra-zdrowia-lista-lekow-refundowanych?fbclid=IwAR1U3YB3yON5EN2s1qdYRbcIeh7iDxqeOtQoEYGFvX9ozGDWdURIK2JOMRs}
{Ministerstwo Zdrowia}
oraz \href{https://www.nfz.gov.pl/aktualnosci/aktualnosci-centrali/komunikat-dgl,7465.html?fbclid=IwAR0F41XjLwTg7XQdUjeYpE_KS4VVZk50etlbYDpwxhxOR2ZLdslMatUtbEU}{NFZ}
w postaci arkuszy Excel.

Użytkownik ma możliwość wyszukiwania leków po ich nazwie, dawce, substancji czynnej, opakowaniu, kodzie EAN, poziomie odpłatności, jednostkach chorobowych. Dodatkowo ma możliwość znalezienia najtańszego spośród identycznych, tj. o takiej samej substancji czynnej oraz dawkowaniu do danego. Dla każdego jest dostępna statystyka opisująca jak pacjenci kupują dany lek oraz informacja o najbliższej aptece, która posiada go w sprzedaży. 

Dane udostępniane przez Ministerstwo niestety czasami są błędne, bądź niespójnie. W celu całkowitej pewności, dzięki zewnętrznym serwisom, jest sprawdzane czy dany lek na pewno powinien być przypisaywany do podanych jednostek chorobowych. Aplikacja pozwala kontrolować istniejące pomyłki, posługuje się jedynie poprawnymi, a użytkownik ma możliwość przejrzenia pomyłek w celu ich weryfikacji bądź zgłoszenia twórcom danych.


- Ranking lekow
- Dawkowanie leków

HUJ HUJ HUJ HUJ

\section{Podsumowanie}

Dzięki udostępnianiu informacji o refundacjach społeczeństwo może dowiedzieć się jakie leki w aptekach będzie można nabyć w niższych cenach. Jednakże sposób upubliczniania tych informacji uniemożliwia swobodne przeglądanie i analizowanie substancji.


\end{document}
